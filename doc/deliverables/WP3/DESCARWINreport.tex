\documentclass[english]{DESCARWINreport}

\usepackage{amsmath}
\usepackage{amsfonts}
\usepackage{color}
\usepackage{algorithm}
\usepackage[noend]{algorithmic}
%\usepackage{algorithmic}
\usepackage{subfigure}
\usepackage{hyperref}
\hypersetup{colorlinks=true,linkcolor=black}
\usepackage{lscape}
\title{DESCARWIN\\\bigskip {\em \LARGE The Marriage of Descartes and Darwin}}
%ANR-09-COSI-002
%\author{Pierre Sav�ant and Johann Dr�o}
\date{\today}
\laboratory{TRT - INRIA - ONERA}
\docref{62 441 217-306}
\revision{-}

\setlength{\parindent}{0cm}
\setlength{\parskip}{2ex plus 0.5ex minus 0.2ex}

\newcounter{hyp}
\setcounter{hyp}{1}
\newcommand{\hyp}{H\thehyp\stepcounter{hyp}}
\newcounter{defi}
\setcounter{defi}{1}
\newcommand{\defi}{D\thedefi\stepcounter{defi}}
\newcounter{con}
\setcounter{con}{1}
\newcommand{\con}{C\thecon\stepcounter{con}}

% Pour r�duire globalement l'espace entre les items d'une liste
% on peut �galement utiliser le bout de code suivant de M. Wooding
% Les param�tres utilis�s pour d�finir cette mise en page
% sont les suivants :
% \topsep espace vertical suppl�mentaire (ajoute � \parskip)
% 	ins�r� entre le texte pr�c�dant la liste et le 1er objet
% 	de la liste
% \partosep espace vertical suppl�mentaire ins�r� devant la liste
% 	si celle-ci est pr�c�d�e d'une ligne blanche
% \itemsep espace vertical suppl�mentaire (ajout� � \parsep)
% 	ins�r� entre les �l�ments d'une liste.

%%%% debut macro %%%%
\makeatletter
\toks@\expandafter{\@listI}
\edef\@listI{\the\toks@\setlength{\parsep}{0pt}}
\edef\@listI{\the\toks@\setlength{\topsep}{0pt}}
\makeatother
%%%% fin macro %%%%

\begin{document}

\maketitle

%\cleardoublepage

\begin{revisions}
\begin{revtable}
\dates{APR. 1, 2011}{}{}{}{}
\writers{Author1\\Author2}{}{}{}{}
\approvers{P. Sav\'eant}{}{}{}{}
\end{revtable}
\begin{revisionlabels}
\revlabel{initial version}
\revlabel{}
\end{revisionlabels}
\end{revisions}

%\begin{figure}[htbp]
%\vspace{-0.5cm}
%\centering
%\includegraphics[width=0.25\textwidth]{Salon_du_Bourget_20090619_114_GroundSearch_1000km.jpg}
%\end{figure}

\begin{abstract}
The object of this document is to provide the formal specification of the configuration problem to be solved.
\end{abstract}

%\begin{figure}[htbp]
%\centering
%\includegraphics[width=0.70\textwidth]{../Images/Salon_du_Bourget_20090619_114_GroundSearch_1000km.jpg}
%\end{figure}

\tableofcontents

\newpage

\chapter{Intoduction}


\chapter{Overview}


\chapter{XXX}


\chapter{YYY}


\chapter{ZZZ}

\section{zzz}

\appendix

\chapter*{Appendix}

\section*{examples}

\subsection*{EXAMPLE 1}

\subsection*{EXAMPLE 2}

\end{document}
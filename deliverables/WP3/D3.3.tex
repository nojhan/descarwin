\documentclass[english]{DESCARWINreport}

\usepackage{amsmath}
\usepackage{amsfonts}
\usepackage{color}
\usepackage{algorithm}
\usepackage[noend]{algorithmic}
%\usepackage{algorithmic}
\usepackage{subfigure}
\usepackage{hyperref}
\hypersetup{colorlinks=true,linkcolor=black}
\usepackage{lscape}
\title{DESCARWIN\\\bigskip {\em \LARGE The Marriage of Descartes and Darwin}\\\vspace{8cm} 
{\LARGE D3.3\\
Multi-objective Experimentations}}
%ANR-09-COSI-002
%\author{Pierre Savéant and Johann Dréo}
\date{\today}
\laboratory{TRT - INRIA - ONERA}
\docref{62 441 217-306}
\revision{-}

\setlength{\parindent}{0cm}
\setlength{\parskip}{2ex plus 0.5ex minus 0.2ex}

\newcounter{hyp}
\setcounter{hyp}{1}
\newcommand{\hyp}{H\thehyp\stepcounter{hyp}}
\newcounter{defi}
\setcounter{defi}{1}
\newcommand{\defi}{D\thedefi\stepcounter{defi}}
\newcounter{con}
\setcounter{con}{1}
\newcommand{\con}{C\thecon\stepcounter{con}}

% Pour réduire globalement l'espace entre les items d'une liste
% on peut également utiliser le bout de code suivant de M. Wooding
% Les paramètres utilisés pour définir cette mise en page
% sont les suivants :
% \topsep espace vertical supplémentaire (ajoute à \parskip)
% 	inséré entre le texte précédant la liste et le 1er objet
% 	de la liste
% \partosep espace vertical supplémentaire inséré devant la liste
% 	si celle-ci est précédée d'une ligne blanche
% \itemsep espace vertical supplémentaire (ajouté à \parsep)
% 	inséré entre les éléments d'une liste.

%%%% debut macro %%%%
\makeatletter
\toks@\expandafter{\@listI}
% \edef\@listI{\the\toks@\setlength{\parsep}{0pt}}
% \edef\@listI{\the\toks@\setlength{\topsep}{0pt}}
\makeatother
%%%% fin macro %%%%

\usepackage[final]{pdfpages}

\hoffset -2cm
\textwidth 15cm

\newcommand{\DAE}{{\sc DaE}}
\newcommand{\DAEYAHSP}{{\sc DaE$_{\text{YAHSP}}$}}

\begin{document}

\maketitle

%\cleardoublepage

\begin{revisions}
\begin{revtable}
\dates{May 7., 2013}{}{}{}{}
\writers{Johann Dr\'eo\\Mostepha Khouadjia\\Pierre Savéant\\Marc Schoenauer\\Vincent Vidal}{}{}{}
\approvers{P. Sav\'eant}{}{}{}
\end{revtable}
\begin{revisionlabels}
\revlabel{initial version}
\revlabel{}
\end{revisionlabels}
\end{revisions}

%\begin{figure}[htbp]
%\vspace{-0.5cm}
%\centering
%\includegraphics[width=0.25\textwidth]{Salon_du_Bourget_20090619_114_GroundSearch_1000km.jpg}
%\end{figure}

\begin{abstract}
The object of this document is to provide results of the experimentation campaign regarding the original multi-objective approach to AI planning developed in the Descarwin project. It is made of 3 chapters, corresponding to 3 publications describing the different steps of our experimental campaign that validate our approach to Pareto-based multi-objective AI planning. Each chapter starts with an extended abstract that introduces the context of the work, followed by a copy of the published paper.

\begin{itemize}
 \item This first validation was performed on the original {\tt Zeno} benchmark suite, as described in Deliverable 3.1. The benchmark is first described in detail, and the experimental parts compare several multi-objective evolutionary engines in several instances of this benchmark.

This work has been published as: ``Mostepha Redouane Khouadjia, Marc Schoenauer, Vincent Vidal, Johann Dréo and Pierre Savéant, {\em Multi-Objective AI Planning: Evaluating DaEYAHSP on a Tunable Benchmark}. In Robin C. Purshouse, Peter J. Fleming, and Carlos M. Fonseca, Eds, Proc. 7th International Conference on Evolutionary Multi-Criterion Optimization (EMO2013), pp 36-50, LNCS 7811, Springer Verlag, 2013.'' 

\item This second validation compares the Pareto-based approach that was found the best-performing in the previous chapter, i.e. the one that uses IBEA (the indicator-based evolutionary multi-objective algorithm) with hypervolume indicator, with the more traditional aggregation-based approach using the single objective version of \DAE. The 

This work has been published as: ``Mostepha Redouane Khouadjia, Marc Schoenauer, Vincent Vidal, Johann Dréo and Pierre Savéant, {\em Multi-Objective AI Planning: Comparing Aggregation and Pareto Approaches}. In Martin Middendorf and Christian Blum, Eds, Proc. 13th European Conference on Evolutionary Computation in Combinatorial Optimisation (EvoCOP2013), pp 202-213, LNCS 7832, Springer Verlag, 2013.''

\item This last series of experiments validates the multi-objective \DAEYAHSP\ approach against the only competitor from the AI Planning community, i.e. the approach using the metric-sensitive planner LPG, one of the state-of-the-art planner in single-objective setting, that can however directly handle aggregated objectives. The comparative experiments here involve not only the {\tt Zeno} test-bench, but also the other domains and instances proposed in Deliverable 3.1 based on the multi-objectivizations of some IPC7 (single-objective) domains. In most cases, \DAEYAHSP\ is found to outperform the LPG-based approach.

This work has been accepted at IJCAI 2013 conference, August 2013, as ``Mostepha Redouane Khouadjia, Marc Schoenauer, Vincent Vidal, Johann Dréo and Pierre Savéant, {\em Pareto-Based Multiobjective AI Planning}. In Proc. 23rd International Joint Conference on Artificial Intelligence (IJCAI 2013), 2013.''
\end{itemize}



\end{abstract}

%\begin{figure}[htbp]
%\centering
%\includegraphics[width=0.70\textwidth]{../Images/Salon_du_Bourget_20090619_114_GroundSearch_1000km.jpg}
%\end{figure}

\tableofcontents

\newpage

\chapter{Evaluating \DAEYAHSP\ on a Tunable Benchmark}



\newpage
\hoffset 0cm
\includepdf[height=32cm,pages=-,offset=0cm -4cm]{emo2012.pdf}
\newpage
\hoffset -2cm

\chapter{Comparing Aggregation and Pareto Approaches}

\newpage
\hoffset 0cm
\includepdf[height=32cm,pages=-,offset=0cm -4cm]{evocop2012.pdf}
\newpage
\hoffset -2cm

\chapter{Comparing Metric-Sensitive and Pareto Approaches}

\newpage
\hoffset 0cm
\includepdf[width=22.5cm,height=32cm,pages=-,offset=0cm -4cm]{567.pdf}
\hoffset -2cm


\end{document}